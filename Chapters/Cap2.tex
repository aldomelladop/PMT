\chapter{Marco Teórico}
% ***********************************************************************************

%En linguística existe un esquema de comunicación, llamado \textit{''Esquema de Comunicación de Jakobson''} en el que se enuncian los elementos que forman parte de un proceso de comunicación y que lo conforman; el emisor, el receptor, el código, el mensaje y el canal, los cuales son todos elementos presentes en un proceso de comunicación como el que existen hoy en día, donde hablamos desde actos comunicativos más tradicionales como una nota, una conversación, hasta otros más recientemente instalados en nuestro día a día como lo son el enviar un email, un mensaje de texto, un mensaje por Facebook o Whatsapp. Para que esto sea posible, además de existir un emisor, un receptor y un mensaje, debe existir un canal, y en comunicaciones existen dos categorías que agrupan los tipos de medios, estos pueden ser guiados o no guiados.\\
%
%En el caso de las telecomunicaciones se tiene que el canal de transmisión es el aire, y que por el las ondas electromagnéticas, se propagan sujetas a atenuaciones, reflexiones y refracciones que se producen en las distintas superficies en las que estas inciden.
%%Los medios guiados corresponden a medios de transmisión tales como el cable de par trenzado, cable coaxial o fibra óptica. Mientras en que los medios no guiados son aquellos que utilizan la energía electromagnética que viaja en forma de ondas en un amplio espectro de longitudes de onda, para hacer uso de ella en tecnologías inalámbricas de transmisión de datos en las que se puede encontrar al \ac{WiFi}.\\
%%
%%La tecnología \ac{WiFi} permite la interconexión no cableada de dispositivos electrónicos, permitiendo que estos puedan conectarse a internet a través de un Punto de acceso o \ac{AP}. Este es un dispositivo que opera dentro de lo que el modelo a nivel de la capa física, del modelo OSI.\\

En el proceso de comunicación se tiene que aparte del emisor, el receptor y el mensaje, existe el canal, el cual para la propagación de las ondas electromagnéticas corresponden al aire o bien, a un cable conductor.\\

\textbf{Espectro electromagnético}

Corresponde al conjunto de todas las radiaciones electromagnéticas para las distintas longitudes de onda. Ejemplos de estos tipos de radiación son los rayos gamma, que corresponden a longitudes de onda cortas, pero a altísimas frecuencias.\\

\textbf{Medio No Guiado}\\

Corresponde a los medios por los cual viajan las ondas electromagnéticas y que en este caso, corresponden a dispositivos que se valen de antenas para propagarlas.\\

\textbf{Modelo de propagación}\\

Son aproximaciones matemáticas que permiten modelar el espacio físico por el cual se va a propagar la señal y que contemplan los principios físicos que la describen y que se introducen al modelo en forma de parámetros.

\textbf{Modelo de propagación Indoor}


