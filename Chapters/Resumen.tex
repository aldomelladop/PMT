\chapter*{Resumen}

La historia nos dice que conforme evolucionan las especies también lo hacen sus necesidades. En un comienzo, el humano nómada necesitaba de alimento y cobijo, y debido a esto adaptó sus hábitos errantes para dar paso al asentamiento. Años más tarde, con el desarrollo de la agricultura y el paso hacia el sedentarismo, el ser humano adaptó su entorno para que éste saciara sus requerimientos. Sin embargo, con el paso del tiempo y al evolucionar el colectivo, y por tanto, las estructuras políticas y sociales, el ser humano se vio en la necesidad de volver a movilizarse por motivos diferentes a supervivencia elemental. Volvió a viajar, recorrer nuevos terrenos, y para ello requirió de nuevas herramientas: mapas, cartas de navegación, y más tarde, radares.\\

En la actualidad, para satisfacer estas nuevas necesidades hemos creado sofisticadas y novedosas herramientas para guiarnos, las cuales nos permiten, entre otras cosas, llegar de una ciudad a otra, encontrar un local de comida al otro lado de la cuidad y movilizarnos por la ruta más óptima, una que nos ahorre el tráfico o tediosos pagos de peaje. Todo esto y más ha sido resuelto con la implementación de algoritmos en aplicaciones móviles que nos permiten hacer todo lo anteriormente mencionado.\\

Sin embargo los algoritmos usados en espacios exteriores se valen del \ac{GPS} resultan, dada la precisión de sus mediciones, poco eficiente para determinar la posición en un espacio interior. La señal del satélite no es capaz de atravesar los muros y, es por este motivo que, luego un acucioso estudio, se desarrollará un tipo de \ac{IPS} que permita determinar la ubicación de un objetivo en un espacio indoor.
