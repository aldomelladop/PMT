\chapter*{Resumen}

La historia nos dice que conforme evolucionan la especies lo hacen sus necesidades. En un comienzo el humano nómada necesitaba alimento y cobijo y debido a esto, se adaptó a sus necesidades, años mas tarde con la agricultura y la domesticación precindió de moverse de un lugar a otro para las necesidades básicas e hizo que el entorno se adaptara a sus requerimientos, sin embargo, al evolucionar el colectivo y por tanto, las estructuras políticas y sociales el ser humano tuvo que movilizarse, viajar, recorrer nuevos terrenos y para ello requirió de herramientas; mapas, cartas de navegación y más tarde, radares.\\

En la actualidad, para satisfacer nuevas y complejas necesidades hemos creado sofisticadas y novedosas herramientas para guiarnos, las cuales nos permiten, entre otras cosas llegar de una ciudad a otra, encontrar un local de comida al otro lado de la cuidad y ahorrarnos el tráfico movilizandonos por la ruta más óptima, una que nos ahorre el tráfico o tediosos pagos de peaje. Todo esto y más ha sido resuelto con la implementación de algoritmos en aplicaciones móviles que nos permiten hacer todo lo anteriormente mencionado.\\

Los algoritmos usados en espacios exteriores se valen del \ac{GPS}, sin embargo, este sistema resulta, dada la precisión de sus mediciones poco eficiente a la hora de determinar la posición en un espacio interior, esto pues la señal del satélite no es capaz de atravesar los muros y otras superficies, y es por este motivo que, luego de hacer un estudio respectivo, se implementará un tipo de \ac{IPS} que permita determinar la ubicación de un objetivo al interior de un espacio indoor.
