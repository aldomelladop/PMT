% ***********************************************************************************
% ---------------------            INTRODUCCIÓN                 ---------------------
% ***********************************************************************************
\chapter{Introducción}

% ***********************************************************************************
% ***********************************************************************************
\section{Antecedentes Históricos}

En la década de los 60's el Departamento de Defensa de los Estados Unidos desarrolló el primer prototipo de un sistema de geolocalización basado en el cálculo de la diferencia de fases de la señales emitidas desde un satélite a estaciones terrestres, lo que les permitía determinar su posición. Años más tarde, en la década de los 70's los programas de la armada y de la Fuerza Aérea desarrollaron un sistema mejorado que les permitía determinar posiciones en el planeta mediante el uso de 24 satélites dispuestos a 20.200 km.\\

Mediante el uso de un método matemático llamado Trilateración, son capaces de usar los satélites y la cobertura estos hacen sobre la superficie de la tierra, para comparar la posición del dispositivo, con otras 3 unidades fijas de distancia. Las mediciones y la cobertura de este sistema se hace más exacta conforme se cubren más áreas, de ahí la necesidad de 24 satélites.\\

Sin embargo, dicho método no aplica para espacios indoor y es allí donde aparecen prototipos de GPS para indoors, también llamados Sistemas de Posicionamiento Indoor (\ac{IPS}) en los que implementaciones como las hechas por Google que, tras añadir a su aplicación Google Maps planos de pisos en centros comerciales, aeropuertos y otras grandes áreas comerciales, permiten al usuario moverse estando en conocimiento de accesos, salidas de emergencia y ubicaciones de tiendas. 


% ***********************************************************************************
% ***********************************************************************************
\section{Definición del Problema}

Para personas con algún tipo de discapacidad es de por sí muy tedioso y dificil hallar un lugar en el estacionamiento, encontrar un acceso habilitado para sillas de ruedas, ascensores y baños, especialmente si se trata de algún usuario que se halla de paso en la ciudad.\\

Además de lo anterior, es sabido que para las tiendas de retail es importante poder concentrar de mejor manera sus esfuerzos para captar clientes y comprender el comportamiento basado en el interés que estos muestran por determinados productos y la forma en que la disposición de estos influye en sus hábitos de compra.\\

Todo esto se puede lograr diseñando un sistema de posicionamiento indoor que entregue una ubicación exacta al usuario, ya sea para encontrar accesos y facilitar la navegación de este al interior de un centro comercial, de un hospital o para guiar un robot, para hallar un vehículo en un estacionamiento subterráneo o para entregar informes sobre comportamientos de clientes en una tienda en particular.\\

La utilidad y aplicabilidad de este sistema incuestionable\footnote{https://www.prnewswire.com/news-releases/indoor-positioning-and-navigation-system-market-to-provide-over-usd-25-billion-revenue-post-2016-300362071.html} y es por ello que urge una implementación que de respuesta a esta oportunidad.

% ***********************************************************************************
% ***********************************************************************************
\section{Estado del Arte}

En la actualidad existen diversos algoritmos de posicionamiento indoor basados en el uso de redes Wifi, no obstante, pueden ser clasificados en categorías según el tipo de enfoque que tienen estos: Algoritmo de proximidad, algoritmo de triangulación y algoritmo de análisis de ambiente.



% ***********************************************************************************
% ***********************************************************************************
\section{Hipótesis de Trabajo}
En lenguaje coloquial, es con lo que uno se ``compromete''.

% ***********************************************************************************
% ***********************************************************************************
\section{Objetivos}
% ***********************************************************************************
\subsection{Objetivo General}


% ***********************************************************************************
\subsection{Objetivos Específicos}
\begin{itemize}
 \item Objetivo 1
 \item Objetivo 2
 \item Objevivo 3..
\end{itemize}

% ***********************************************************************************
% ***********************************************************************************
\section{Alcances y Limitaciones}
Aquí se indican los aportes mayores realizados por este trabajo y se indican claramente las limitaciones asumidas.

En general las limitaciones se dejan planteadas a resolver en el trabajo futuro.


% ***********************************************************************************
% ***********************************************************************************
\section{Temario y Metodología}
Se hace una pequeña descripción del contenido de cada capítulo.