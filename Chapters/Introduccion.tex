% ***********************************************************************************
% ---------------------            INTRODUCCIÓN                 ---------------------
% ***********************************************************************************
\chapter{Introducción}

% ***********************************************************************************
% ***********************************************************************************
\section{Antecedentes Históricos}

En la década de los 60's, el Departamento de Defensa de los Estados Unidos desarrolló el primer prototipo de un sistema de geolocalización basado en el cálculo de la diferencia de fases de la señales emitidas desde un satélite a estaciones terrestres, lo que les permitía determinar su posición. Años más tarde, en la década de los 70's, los programas de la armada y de la Fuerza Aérea desarrollaron un sistema que les permitía determinar posiciones mediante el uso de 24 satélites dispuestos a una órbita a 20.200 km sobre la superficie terrestre.\\

Mediante el uso del método matemático llamado Trilateración, usan los satélites, y la cobertura que estos hacen sobre la superficie de la tierra, para comparar la posición del dispositivo, respecto de otras 3 unidades fijas de distancia. Las mediciones y la cobertura de este sistema se hacen más exactas conforme se cubre más áreas, de ahí la necesidad de 24 satélites.\\

Sin embargo, el uso del GPS no aplica para espacios interiores, por este motivo aparecen prototipos de GPS para indoors, también llamados Sistemas de Posicionamiento Interior o (\ac{IPS}) en los que implementaciones recientes como las hechas por Google que, tras añadir a su aplicación Google Maps planos de pisos en centros comerciales, aeropuertos y otras grandes áreas comerciales, permiten al usuario moverse estando en conocimiento de accesos, salidas de emergencia y ubicaciones de tiendas. 

% ***********************************************************************************
% ***********************************************************************************
\section{Definición del Problema}


Para personas con algún tipo de discapacidad es de por sí muy tedioso y difícil hallar un lugar en el estacionamiento, encontrar un acceso habilitado para sillas de ruedas, ascensores y baños, especialmente si se trata de algún usuario que se halla de paso en la ciudad.\\

Además de lo anterior, es sabido que para las tiendas de retail es importante poder concentrar de mejor manera sus esfuerzos para captar clientes y comprender el comportamiento basado en el interés que estos muestran por determinados productos y la forma en que la disposición de estos influye en sus hábitos de compra.\\

Todo esto se puede lograr diseñando un sistema de posicionamiento indoor que entregue una ubicación exacta al usuario, ya sea para encontrar accesos y facilitar la navegación de este al interior de un centro comercial, de un hospital o para guiar un robot, para hallar un vehículo en un estacionamiento subterráneo o para entregar informes estadísticas sobre las trayectorias de los clientes alrededor de una tienda en particular.\\

La utilidad y aplicabilidad de este sistema, tal como lo menciona un informe  sobre la inversión realizada en tecnologías de posicionamiento indoor es incuestionable y es por ello que urge una implementación que de respuesta a esta oportunidad.

%La utilidad y aplicabilidad de este sistema, tal como lo menciona un informe  sobre la inversión realizada en tecnologías de posicionamiento indoor\footnote{https://www.prnewswire.com/news-releases/indoor-positioning-and-navigation-system-market-to-provide-over-usd-25-billion-revenue-post-2016-300362071.html}, es incuestionable y es por ello que urge una implementación que de respuesta a esta oportunidad.

% ***********************************************************************************
% ***********************************************************************************
\section{Estado del Arte}

\underline{\textit{Trilateración}}\\

En abril del año 2014, en International Journal of Scientific and Engineering Research se publicó el artículo "Wifi Indoor Positioning System Based on RSSI Measurements from Wi-fi Access Points - A Tri-lateration Approach", en que Pratik Palaskar y Onkar Pathak, entre otros autores hablan de un algoritmo para obtener la posición de un usuario al interior de un espacio indoor mediante el cálculo de la distancia respecto de tres diferentes puntos de acceso o \ac{AP}, en que además de calcular la distancia, se busca el punto de intersección entre tres circunferencias que representa el área de cobertura del AP. En el además se hace uso de los valores de distancia de propagación que entrega la ecuación de Friis y el modelo de propagación indoor en espacio interior.\\

En el trabajo que hicieron, hablan de este modelo básico de decaimiento de la señal en espacio interior, el cual, está sujeto a las reflexiones, refracciones y atenuaciones y el cual mediante la consideración de los parámetros anteriores y de lo que se puede expresar en la ecuación 1.1

\begin{equation}
P(X) = 10\cdot{n}\cdot{log\left({\dfrac{d}{d_0}}\right)} + 20\cdot{log\left({\dfrac{4\pi d_0}{\lambda}}\right)}  + X_{\sigma}
\end{equation}

Tal que \textit{P(X)}, es la pérdida por ruta, \textit{n}, es el exponente de decaimiento de la señal, \textit{d}, es la distancia entre el transmisor y el receptor, \textit{$d_0$}, la distancia de referencia\footnote{Considerada como 1 m}, \textit{$\lambda$} es la longitud de onda de la señal a $f = 2.4$ GHz. Finalmente se tiene que el  último parámetro, esto es \textit{$X_{\sigma}$} debe calcularse en sitio, esto debido a que varía la cantidad de muros, no obstante, se considera con un valor de $X_{\sigma} = 10$ dB.\\

El modelo de propagación indoor les permitió obtener valores de la distancia entre donde se ubicaba el AP y donde se hallaba el usuario. Esto es fundamental al desarrollar un \ac{IPS} dado que este algoritmo lo que hace es calcular la posición exacta de un usuario dada la posición exacta de un AP, de modo que la poca precisión de estas resultaría en una imprecisión generalizada del algoritmo. \\

\newpage 
Además de lo anterior, se tiene que lo que en realidad es la base de este método, es un sistema de ecuaciones de tres incógnitas que calculan la ubicación espacial del usuario.\\

Al trabajar con la trilateración se prescinde de un seguimiento por dirección MAC, ya que solo se trabajó y aplicó para un dispositivo, sin llegar a una tabulación de resultados para distintos dispositivos trabajando en simultáneo.\\

\underline{\textit{Triangulación}}\\

Por otro lado, existen otras formas, alternativas a la de trilateración, como el que presentan en ''A Comparison of algorithms adopted in fingerprinting indoor positioning systems'' en donde lo que se hace, es basar la estimación de la ubicación del objetivo, en las propiedades geométricas de los triángulos, a diferencia de lo que se presentaba en \cite{1}, en que se proponía hacerlo usando la intersección de las circunferencias que representaban a los \acp{AP}. En esta forma, se ocupan parámetros tales como el \ac{TOA}, el \ac{AOA} y  el \ac{RSS} para pdoer hacer el cálculo. Estos parámetros son entregados directamente por los \acp{AP} y se usan para calcular las distancias entre el objetivo y los AP's, al igual que lo que se plantea en  \cite{1} se debe disponer de a lo menos 3 AP's para poder estimar su ubicación.\\

\underline{\textit{Algoritmos de Proximidad}}\\

En el texto de ''Recent Advances in Wireless Indoor Localization Techniques
and System'' se tiene que este tipo de algoritmos ayuda a estimar la ubicación del objetivo usando la relación de proximidad entre el objetivo y los puntos de acceso Wifi. Esto es, cuando el dispositivo móvil recibe la señal desde diferentes \acp{AP}, la compara a partir de las señales recibidas  se la asigna al AP con la RSS más fuerte. La precisión de este algoritmo está dada por la densidad de distribución y la señal de los AP Wifi.\\

%%Como se puede ver en \cite[pág 21]{dan} ...


% ***********************************************************************************
% ***********************************************************************************
%\section{Hipótesis de Trabajo}


% ***********************************************************************************
% ***********************************************************************************
\section{Objetivos}
% ***********************************************************************************
\subsection{Objetivo General}

Investigar sobre los distintos algoritmos de posicionamiento indoor, para en la memoria de título, desarrollar un sistema que permita estimar la posición de un usuario en un espacio indoor.

% ***********************************************************************************
\subsection{Objetivos Específicos}
\begin{itemize}
 \item {Investigar sobre las distintas alternativas de \ac{IPS} usadas actualmente.}
\item{Estudiar las ventajas y desventajas existentes entre las distintas alternativas de \ac{IPS}}
\item{Estudiar el comportamiento de una red wifi en un espacio indoor apoyado en un modelo de propagación.}
 \item {Estudiar el modelo de pérdidas que se adecúe mejor al espacio sobre el que se harían las mediciones.}
\item{Estudiar antecedentes sobre IPS que utilicen \ac{IRDP}}
\end{itemize}

% ***********************************************************************************
% ***********************************************************************************
\section{Alcances y Limitaciones}
\textit{Alcances}\\

Acotar el estudio a modelos de propagación indoor que se adecuen mejor al espacio que se desea caracterizar para hacer las mediciones.\\
Acotar el estudio a la 
\textit{Limitaciones}\\

Escoger un espacio que sea representativo para realizar las mediciones, es decir, que sea acorde a los espacios en los que podría llegar a implementarse el seguimiento.\\
Disponibilidad de documentos que expongan y exploten la vulnerabilidad  de seguridad que supone el envío de la MAC en las solicitudes IRDP.

% ***********************************************************************************
% ***********************************************************************************
\section{Temario y Metodología}

En el segundo capítulo, se presentarán y abordarán con mayor detalle los modelos de propagación, para así cotejar la validez y concordancia entre los valores de RSS teóricos y los recibidos, además de presentar una alternativa de estudio de IPS como es el caso del seguimiento de dispositivos móviles en base a las solicitudes que éstos envían al router cuando se desean conectar. 