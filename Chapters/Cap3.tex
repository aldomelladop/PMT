\chapter{Revisión Bibliográfica}
% ***********************************************************************************
\section{Introducción}
En este capítulo se hará una revisión de los textos consultados en el desarrollo de este estudio, ahondando en los elementos que estos aportan al definir, de entre todos los métodos, modelos y técnicas de posicionamiento Indoor.\\
% ***********************************************************************************
\subsection{Recent Advances in Wireless Indoor Localization Techniques and System}
\subsubsection{Abstract}

\emph{Los avances en las tecnologías basadas en la localización y la creciente importancia de la informática ubicua y la información dependiente del contexto han llevado a un creciente interés comercial en aplicaciones y servicios basados en la ubicación. En la actualidad, la mayoría de los requisitos de las aplicaciones son la localización o el seguimiento en tiempo real de las pertenencias físicas dentro de los edificios con precisión; por lo tanto, la demanda de servicios de localización en interiores se ha convertido en un requisito previo clave en algunos mercados. Además, las tecnologías de localización de interiores abordan la inadecuación del sistema de posicionamiento global dentro de un entorno cerrado, como los edificios. Sin embargo, sobre la base de esto, este documento tiene como objetivo proporcionar al lector una revisión de los avances recientes en técnicas y sistemas de localización de interiores inalámbricos para ofrecer una mejor comprensión del estado del arte de tecnologías y motivar nuevos esfuerzos de investigación en este campo prometedor. Para este propósito, se revisan los sistemas existentes de localización inalámbrica y los esquemas de estimación de ubicación, ya que también comparamos las técnicas y sistemas relacionados con una conclusión y trabajos futuros.}\\

Este documento hace una revisión acabada de los distintos métodos empleados en el desarrollo de sistemas de posicionamiento indoor comparando las ventajas y desventajas entre ellos. Es especialmente útil pues permite entender porqué es que se usan preferentemente métodos asociados a variables que no dependan tanto de la potencia, sino que a variables relacionadas con la propagación de la señal en estos espacios. Es así que se descartan métodos tales  como el de proximidad, aún cuando este tiene un costo computacional bajo, para escoger métodos  relacionados con el ToA, donde se prioriza la exactitud por sobre el costo computacional. Además, se mencionan otros métodos tales como, Map Matching, el cual es un algoritmo basado en algoritmos de proyección y reconocimiento de patrones.\\

Sin embargo, es finalmente la capacidad computacional, así como la inmediatez con la que se necesitan los resultados, lo que sugiere que métodos de bajo requerimiento computacional sean candidatos óptimos.

\subsection{WIFI-Based Indoor Positioning System}
\subsubsection{Abstract}

\emph{El fácil acceso y la disponibilidad de las tecnologías inalámbricas y la informática móvil e Internet han dado lugar a nuevas oportunidades en el desarrollo de aplicaciones móviles cuyo propósito es facilitar la vida de las personas. Hoy en día, una persona puede poseer más de un dispositivo móvil para diferentes usos, como comunicación, entretenimiento, trabajos de oficina. Este documento propone una aplicación móvil que podrá estimar la posición de un usuario dentro de un edificio mediante el uso de la tecnología WIFI.}\\

Este texto plantea que dentro de las posibilidades de desarrollo que se han abordado para poder llegar a implementar un sistema de posicionamiento indoor, han sido consideradas varias tecnologías entre las cuales están; bluetooth, Wifi, RFID e incluso Infrarrojo. Sin embargo, comenta que dada la penetración que ha tenido la tecnología Wifi en ambientes indoor este ha sido elegido por sobre los otros, dejando así, una amplia oportunidad de desarrollo basada en las propiedades de funcionamiento, dado que no requiere invertir en hardware especializado. Razón por la cual se valida usar en este estudio, a Wifi como la tecnología apropiada.\\

Además, este paper introduce nociones sobre lo que es la detección de \ac{RSS} desde los distintos AP disponibles y que, basados en la asociación de las distintas posiciones recibidas e informadas a la dirección MAC, se puede hacer un seguimiento de los dispositivos basándose en la potencia informada y la MAC. Se apoya además de una base de datos que consiste en los niveles de potencia recibidos y almacenados en un vector que caracteriza el espacio, y se complementa con la distancia euclidiana que, como plantean el documento <<\textit{es usada para comparar la posición obtenida actual en la aplicación móvil y la posición almacenada en la base de datos en el servidor.}>>

\subsection{A Comparison of algorithms adopted in figerprint indoor positioning systems}
\subsubsection{Abstract}

\emph{La tecnología de huellas digitales se ha utilizado ampliamente en sistemas de posicionamiento en interiores como los sistemas de posicionamiento Wi-Fi. Su desempeño depende no sólo de la medición de la intensidad de la señal, sino que también del algoritmo utilizado. En este documento, da una visión general de los algoritmos populares actuales adoptados en sistemas de posicionamiento indoor basados en WIFI, incluido el método determinista (K vecino más cercano, K peso vecino más cercano), método probabilístico y red neuronal. En orden para obtener un resultado confiable y representativo, esos algoritmos se evalúan en función en la misma base de datos Se hicieron comparaciones con respecto al posicionamiento precisión, complejidad computacional y el tamaño de la base de datos. Además, los detalles de elección de parámetros y la implementación de estos algoritmos son discutido.}\\

Este texto sirve de apoyo para lo que se planteó en el marco teórico, puntualmente en el item de métodos de estimación de posición basado en Redes Neuronales \ref{redesneuronales}, pues realiza una comparación entre la eficiencia, la calidad, cantidad y naturaleza de los parámetros que se obtienen, decretándose que para el caso de las redes neuronales, si bien se conserva la idea de recabar información desde los niveles de potencia recibidos,  se requiere de un entrenamiento previo tedioso, lento y de alto consumo computacional. Razón por la cual, al requerirse para el tipo de espacio, ambiente y potencial cliente respuestas rápidas las redes neuronales no son una alternativa en el corto plazo.\\

Además, el texto hace una comparación entre otros métodos de estimación de posición basándose en técnicas que se apoyan de la \ac{RSS} y aquellos basados en propiedades geométricas. Del texto se extrae de mejor manera la noción de qué tipo de método debería escogerse, si uno probabilístico o uno empírico.

\subsection{Wi-Fi In-door Positioning System Based on RSSI Measurements from Wi-Fi Access Points - A Tri-lateration Approach}
\subsubsection{Abstract}	
%Positioning is the most attractive technology today. Various technologies are used now days for positioning purpose. GPS is mainly used for outdoor environment. Non-suitability of GPS in indoor conditions because of its NLOS conditions and signal attenuation has lead to several other techniques of indoor positioning. This paper compares few indoor positioning methods and proposes indoor positioning system using tri-lateration method which uses RSSI data from wi-fi access points to do localization in indoor environment.\\

\emph{El posicionamiento es la tecnología más atractiva hoy en día. Varias tecnologías se utilizan ahora días para el propósito de posicionamiento. El GPS se usa principalmente para entornos al aire libre. La no idoneidad del GPS en interiores debido a sus condiciones NLOS y la atenuación de la señal ha llevado a varias otras técnicas de posicionamiento en interiores. Este documento compara pocos métodos de posicionamiento en interiores y propone un sistema de posicionamiento en el interior que utiliza un método de trilateración que utiliza datos RSSI de los puntos de acceso wi-fi para hacer la localización en el ambiente interior.}\\

Además de los métodos basados en parámetros no relacionados con la potencia, existen aquellos que combinan propiedades geométricas y nociones implementadas en tecnologías predecesoras a los IPS, para dar como resultado un método de trilateración que experimentalmente demuestra tener resultados de calidad mas bien baja, y que si bien, los resultados permiten estimar la posición, distan considerablemente del resultado que se persigue en este estudio, mas entregan un marco de referencia, tanto en conocimientos teóricos y de resultados experimentales que ciertamente serán de utilidad a la hora de optar por un método u otro.


\subsection{WIFI-based indoor positioning system}
\subsubsection{Abstract}	
%Recently, several indoor localization solutions based on WiFi, Bluetooth, and UWB have been proposed. Due to the limitation and complexity of the indoor environment, the solution to achieve a low-cost and accurate positioning system remains open. This article presents a WiFi- based positioning technique that can improve the localization performance from the bottle- neck in ToA/AoA. Unlike the traditional approaches, our proposed mechanism relaxes 
%the need for wide signal bandwidth and large numbers of antennas by utilizing the transmis- sion of multiple predefined messages while maintaining high-accuracy performance. The overall system structure is demonstrated by showing localization performance with respect to different numbers of messages used in 20/40 MHz bandwidth WiFi APs. Simulation resultsshow that our WiFi-based positioning approach can achieve 1 m accuracy without any hardware change incommercial WiFi products, which is much better than the conventional solutions from both academia and industry concerning the trade-off of cost and system complexit.

\emph{Recientemente, se han propuesto varias soluciones de localización de interiores basadas en WiFi, Bluetooth y UWB. Debido a la limitación y complejidad del entorno interior, la solución para lograr un sistema de posicionamiento preciso y de bajo costo permanece abierta. Este artículo presenta una técnica de posicionamiento WiFibased que puede mejorar el rendimiento de localización del cuello de botella en ToA / AoA. A diferencia de los enfoques tradicionales, nuestro mecanismo propuesto relaja la necesidad de un amplio ancho de banda de señal y un gran número de antenas al utilizar la transmisión de múltiples mensajes predefinidos mientras se mantiene el rendimiento de alta precisión. La estructura general del sistema se demuestra al mostrar el rendimiento de localización con respecto a los diferentes números de mensajes utilizados en los AP WiFi de banda ancha de 20/40 MHz. Los resultados de simulación muestran que nuestro enfoque de posicionamiento basado en WiFi puede alcanzar 1 m de precisión sin ningún cambio de hardware en productos WiFi comerciales, que es mucho mejor que las soluciones convencionales de la academia y la industria en relación con el costo y la complejidad del sistema.}\\

La utilidad de este documento reside en la información que entrega respecto de la calidad y naturaleza de los parámetros utilizados, donde a diferencia de otros documentos consultados, este presenta variables que no se apoyan sólo de métodos geométricos, ni basados en la intensidad de potencia recibida, sino que entregan herramientas tales como el \ac{TOA}, el \ac{AOA} y un híbrido entre los dos anteriores. La calidad y claridad de los conceptos que entregan permite comprender porqué, como se ha señalado en documentos anteriores, estos parámetros de no ser que estos requieren un alto consumo computacional, serían la base de los IPS. Presenta en este estudio, procedimientos probabilísticos para dotar de mayor precisión a la estimación, herramientas basadas en análisis espectrales de las señales recibidas y que relacionan la fase.

%\subsection{Mobile indoor positioning using Wifi localization and image processing}
%\subsubsection{Abstract}	
%%At present, there has been an increased interest in indoor positioning systems that propelled researchers to come up with various solutions. A number of these research either use Wi-fi Localization or image processing; each having problems with either accuracy or speed. In this paper, we propose a framework which will use both Wi-fi Localization and image processing to address this problem. The framework used Wi-fi Localization to calculate the estimated position of the user then refine the accuracy through image processing. The designed framework was able to surpass the performance of Wi-fi Localization algorithms in accuracy, and image processing in speed. Different techniques were also applied that improved accuracy and made the system calculate for the location faster.
%
%\emph{En la actualidad, ha aumentado el interés en los sistemas de posicionamiento en interiores que impulsaron a los investigadores a encontrar varias soluciones. Algunas de estas investigaciones utilizan la localización Wi-Fi o el procesamiento de imágenes; cada uno tiene problemas con la precisión o la velocidad. En este documento, proponemos un marco que utilizará tanto la localización Wi-Fi como el procesamiento de imágenes para abordar este problema. El marco utilizó la localización Wi-Fi para calcular la posición estimada del usuario y luego refinar la precisión mediante el procesamiento de imágenes. El marco diseñado fue capaz de superar el rendimiento de los algoritmos de localización de Wi-Fi en precisión, y el procesamiento de imágenes en velocidad. También se aplicaron diferentes técnicas que mejoraron la precisión e hicieron que el sistema calculara la ubicación más rápido.}


\subsection{Mobile indoor positioning using Wifi localization}
\subsubsection{Abstract}
\emph{El objetivo de este documento es crear una aplicación que describa el método de trilateración Wi-Fi para el posicionamiento en interiores utilizando dispositivos móviles basados en Android. Los objetivos, alcances, limitaciones y la importancia de esta investigación también son temas tratados en este artículo. El problema de propagación de la señal interior se resuelve al recibir una colección de medidas de intensidad de la señal que mejora la precisión de la localización. El tema de precisión implica el conocido problema de la propagación de la señal en interiores que se resuelve mediante la recopilación de mediciones de la intensidad de la señal recibida y su uso en el algoritmo de trilateración. La técnica de posicionamiento en interiores abre posibilidades para el desarrollo de varios sistemas inteligentes que proporcionan al usuario información basada en la ubicación dentro de los edificios.}\\

El aporte de este documento recae en la presentación de una implementación en un sistema operativo android de lo que puede ser un sistema de posicionamiento, plantea que usando el método geométrico de trilateración es posible estimar la posición basándose en la potencia informada por el usuario, lo que permitiría, a través de esta misma aplicación darle a conocer a un usuario su ubicación. Mas dista de ser lo que persigue este estudio dado que si bien resuelve el posicionamiento, es pobremente aplicable a un sistema de seguimiento o de posicionamiento en tiempo real.

\subsection{An Improved WiFi Indoor Positioning Algorithm by Weighted Fusion}
\subsubsection{Abstract}
\emph{El rápido desarrollo de Internet móvil ha brindado la oportunidad de que el posicionamiento en interiores de WiFi pase a ser el centro de atención debido a su bajo costo. Sin embargo, hoy en día la precisión del posicionamiento WiFi en interiores no puede satisfacer las demandas de aplicaciones prácticas. Para resolver este problema, este documento propone un algoritmo de posicionamiento de WiFi mejorado mediante fusión ponderada. El algoritmo propuesto se basa en los algoritmos tradicionales de huellas digitales de ubicación y consta de dos etapas: la adquisición fuera de línea y el posicionamiento en línea. El proceso de adquisición fuera de línea selecciona los parámetros óptimos para completar la adquisición de la señal, y forma una base de datos de huellas dactilares por clasificación y manejo de errores. Para mejorar aún más la precisión del posicionamiento, el proceso de posicionamiento en línea primero usa un método previo al partido para seleccionar las huellas dactilares candidatas para acortar el tiempo de posicionamiento. Después de eso, utiliza la distancia euclidiana mejorada y la probabilidad conjunta mejorada para calcular dos resultados intermedios, y calcula además el resultado final de estos dos resultados intermedios por fusión ponderada. La distancia Euclidiana mejorada introduce la desviación estándar de la intensidad de la señal WiFi para suavizar la fluctuación de la señal WiFi y la probabilidad conjunta mejorada introduce el cálculo logarítmico para reducir la diferencia entre los valores de probabilidad. Comparando el algoritmo propuesto, el algoritmo euclidiano WKNN basado en la distancia y el algoritmo de probabilidad conjunta, los resultados experimentales indican que el algoritmo propuesto tiene una mayor precisión de posicionamiento}\\

Dentro de los documentos consultados, donde se han expuesto métodos basados en potencia, en propiedades geométricas y en el conocimiento claro de la ubicación de los AP, aparece lo que expone el autor, donde además de presentar conceptos esenciales relacionados con los distintos métodos de cálculo de la posición en los que se basa la implementación de los IPS, aparece también la idea de realizar posicionamiento indoor prescindiendo de conocer la ubicación de los APs. Señala que dentro de los problemas que existen a la hora de implementar un sistema, está el hecho de que generalmente no se conocen los APs dispuestos en otros espacios o por otros clientes y además, que la precisión de los métodos presentados anteriormente es mas bien cuestionable debido a ser poco adecuada a los ambientes que se desea implementar.\\

Plantea la necesidad de incluir en el método, un entrenamiento previo, que básicamente recoge los niveles de potencia recibidos en el área de estudio, la cual es previamente dividida en sectores equidistantes y asociar estos niveles al sector en que se halla el usuario. Corrige impresiciones en las mediciones mediante la inclusión de otros parámetros tales como la potencia promedio, la desviación estándar y  el valor promedio de la potencia procesada. Lo destacable de este documento es que expone la posibilidad de estimar la posición del usuario sin tener que conocer la ubicación de los APs.

\subsection{Characterizing Wi-Fi Network Discovery}
\subsubsection{Abstract}
%In the Wi-Fi network discovery phase, the probe request sent by a mobile device is not encrypted. Therefore, the MAC address contained in the probe request frame can be exploited to track users without their knowledge and consent. To counter this problem, modern operating systems implement MAC address randomization. However, studies showed that the spoofed MAC address can still be reversed. In addition, studies showed that some devices send the Service Set Identifiers (SSIDs) of the previously connected networks. Such SSID list reveals sensitive personal information such as the locations one has been to. In order to help understand how much the probe requests reveal the behavior or traces of a user, this thesis characterizes the scanning behavior of the mobile device during network discovery. The different scenarios we tested include different numbers of the saved networks, different connection status, and mobile configurations. The characterized parameters include the channel sequence order, the number of probe request frames sent per channel, the interprobe delay time, and the minimum and maximum channel time a device spent on each channel (MinChannelTime, MaxChannelTime). We test five different devices and find that all of them send one or two probe requests per channel. All devices except the Windows PC set the same value for the minimum and maximum channel time, which ranged from 20 ms to 105 ms. Our experiments show that MacBook prioritizes the channels of the saved networks, which reduces the connection time. Our results show that devices exhibit different scanning behavior in different settings. We found that three of the five devices tested used randomized MAC addresses, and almost none of them sent the SSIDs of the previously connected networks. We also observed that, some devices with MAC randomization reveal their true MAC addresses once connected to a network.\\

\emph{En la fase de descubrimiento de la red Wi-Fi, la solicitud de sonda enviada por un dispositivo móvil no está encriptada. Por lo tanto, la dirección MAC contenida en el marco de solicitud de la sonda puede aprovecharse para rastrear usuarios sin su conocimiento y consentimiento. Para contrarrestar este problema, los sistemas operativos modernos implementan aleatorización de direcciones MAC. Sin embargo, los estudios mostraron que la dirección MAC falsificada aún se puede revertir. Además, los estudios mostraron que algunos dispositivos envían los identificadores de conjunto de servicios (SSID) de las redes conectadas anteriormente. Dicha lista de SSID revela información personal confidencial, como las ubicaciones en las que se ha estado. Para ayudar a comprender cuánto revelan las solicitudes de sondeo el comportamiento o las huellas de un usuario, esta tesis caracteriza el comportamiento de exploración del dispositivo móvil durante el descubrimiento de la red. Los diferentes escenarios que probamos incluyen diferentes números de redes guardadas, diferentes estados de conexión y configuraciones móviles. Los parámetros caracterizados incluyen el orden de secuencia de canal, el número de tramas de solicitud de sonda enviadas por canal, el tiempo de retardo entre pruebas y el tiempo de canal mínimo y máximo que un dispositivo pasó en cada canal (MinChannelTime, MaxChannelTime). Probamos cinco dispositivos diferentes y descubrimos que todos ellos envían una o dos solicitudes de prueba por canal. Todos los dispositivos, excepto Windows PC, configuraron el mismo valor para el tiempo de canal mínimo y máximo, que varió de 20 ms a 105 ms. Nuestros experimentos muestran que la MacBook prioriza los canales de las redes guardadas, lo que reduce el tiempo de conexión. Nuestros resultados muestran que los dispositivos muestran diferentes comportamientos de escaneo en diferentes configuraciones. Encontramos que tres de los cinco dispositivos probados usaban direcciones MAC aleatorizadas, y casi ninguno de ellos enviaba los SSID de las redes previamente conectadas. También observamos que algunos dispositivos con aleatorización MAC revelan sus verdaderas direcciones MAC una vez que se conectan a una red.}\\

Dentro del proceso de escaneo, present{ación y conexión de un dispositivo a la red se tiene que, como lo presenta en el documento, cuentan con la dirección MAC contenida en la solicitud enviada al router en forma de texto plano, pudiendo ser intervenida y usada. Además se expone la posibilidad de hacer un seguimiento de las redes a las que ha estado conectada anteriormente a partir del conjunto de identificadores almacenados en una lista. Esta información no sólo da atisbos de aspectos personales sino que también de la ubicación en la que ha estado previamente conectado el dispositivo. \\

En el documento, se exponen los resultados de la caracterización de una red en el marco de la extracción de estos datos sensibles, que fueron recogidos en diversas circunstancias de variaciones de parámetros.\\

Finalmente, habla de las herramientas y alternativas que buscan revertir la vulnerabilidad de este tipo de solicitudes. 


\subsection{I know your MAC Address: Targeted tracking of individual using Wi-Fi}
\subsubsection{abstract}
%This work is about wireless communications technologies embedded in portable devices, namely Wi-Fi, Bluetooth and GSM. Focusing on Wi-Fi, we study the privacy issues and potential missuses that can affect the owners of wireless-enabled portable devices. Wi-Fi enable-devices periodically broadcast in plain-text their unique identifier along with other sensitive information. As a consequence, their owners are vulnerable to a range of privacy breaches such as the tracking of their movement and inference of private information (Cunche et al. in Pervasive Mobile Comput, 2013; Greenstein in Proceedings of the 11th USENIX workshop on hot topics in operating systems, pp 10:1–10:6. USENIX Association, Berkeley, 2007). As serious as those information leakage can be, linking a device with an individual and its real world identity is not a straightforward task. Focusing on this problem, we present a set of attacks that allow an attacker to link a Wi-Fi device to its owner identity. We present two methods that, given an individual of interest, allow identifying the MAC address of its Wi-Fi enabled portable device. Those methods do not require a physical access to the device and can be performed remotely, reducing the risks of being noticed. Finally we present scenarios in which the knowledge of an individual MAC address could be used for mischief.\\

\emph{Este trabajo trata sobre las tecnologías de comunicaciones inalámbricas integradas en dispositivos portátiles, a saber, Wi-Fi, Bluetooth y GSM. Centrándonos en el Wi-Fi, estudiamos los problemas de privacidad y las posibles amenazas que pueden afectar a los propietarios de dispositivos portátiles con capacidad inalámbrica. Los dispositivos habilitadores de Wi-Fi transmiten periódicamente en texto plano su identificador único junto con otra información sensible. Como consecuencia, sus propietarios son vulnerables a una serie de violaciones a la privacidad, como el seguimiento de su movimiento y la inferencia de información privada (Cunche y otros en Pervasive Mobile Comput, 2013; Greenstein en Procedimientos del 11º taller de USENIX sobre temas candentes en sistemas operativos, pp 10: 1-10: 6. USENIX Association, Berkeley, 2007). Tan grave como puede ser la filtración de información, vincular un dispositivo con un individuo y su identidad real no es una tarea sencilla. Al centrarnos en este problema, presentamos un conjunto de ataques que permiten a un atacante vincular un dispositivo Wi-Fi a su identidad de propietario. Presentamos dos métodos que, dado un individuo de interés, permiten identificar la dirección MAC de su dispositivo portátil habilitado con Wi-Fi. Esos métodos no requieren un acceso físico al dispositivo y se pueden realizar de forma remota, reduciendo los riesgos de ser notado. Finalmente, presentamos escenarios en los que el conocimiento de una dirección MAC individual podría usarse para hacer travesuras.}\\


La vulnerabilidad y por otro lado, la explotabilidad de los datos resulta beneficioso o perjudicial según sea el dispositivo detrás del que se halla la persona. Si es el usuario, claramente el exponer datos tan sensibles como las \ac{SSID}, que es una secuencia de octetos que se incluyen en todos los paquetes de una red inalámbrica para identificarlos como parte de esta. Y por otro lado, la dirección MAC, representan una vulnerabilidad que no sólo podría dar pistas del recorrido que hace una persona al rededor de routers inalámbricos, sino que a ojos de otros, es información que podría dar pistas de patrones de conducta o hábitos que exponen a los espiados.\\

Este documento no solo da una breve idea de lo vulnerable que es el usuario ante este tipo de solicitudes, sino que también entrega nociones respecto de lo que son las herramientas que se usan y cómo estas generan nuevas instancias para sacar provecho ilegal.

